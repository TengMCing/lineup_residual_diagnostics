% interactcadsample.tex
% v1.03 - April 2017

\documentclass[]{interact}

\usepackage{epstopdf}% To incorporate .eps illustrations using PDFLaTeX, etc.
\usepackage{subfigure}% Support for small, `sub' figures and tables
%\usepackage[nolists,tablesfirst]{endfloat}% To `separate' figures and tables from text if required

\usepackage{natbib}% Citation support using natbib.sty
\bibpunct[, ]{(}{)}{;}{a}{}{,}% Citation support using natbib.sty
\renewcommand\bibfont{\fontsize{10}{12}\selectfont}% Bibliography support using natbib.sty

\theoremstyle{plain}% Theorem-like structures provided by amsthm.sty
\newtheorem{theorem}{Theorem}[section]
\newtheorem{lemma}[theorem]{Lemma}
\newtheorem{corollary}[theorem]{Corollary}
\newtheorem{proposition}[theorem]{Proposition}

\theoremstyle{definition}
\newtheorem{definition}[theorem]{Definition}
\newtheorem{example}[theorem]{Example}

\theoremstyle{remark}
\newtheorem{remark}{Remark}
\newtheorem{notation}{Notation}


% tightlist command for lists without linebreak
\providecommand{\tightlist}{%
  \setlength{\itemsep}{0pt}\setlength{\parskip}{0pt}}



\usepackage{lscape}
\usepackage{hyperref}
\usepackage[utf8]{inputenc}
\def\tightlist{}

\usepackage{booktabs}
\usepackage{longtable}
\usepackage{array}
\usepackage{multirow}
\usepackage{wrapfig}
\usepackage{float}
\usepackage{colortbl}
\usepackage{pdflscape}
\usepackage{tabu}
\usepackage{threeparttable}
\usepackage{threeparttablex}
\usepackage[normalem]{ulem}
\usepackage{makecell}
\usepackage{xcolor}

\begin{document}


\articletype{ARTICLE TEMPLATE}

\title{A Plot is Worth a Thousand Tests: Assessing Residual Diagnostics
with the Lineup Protocol}


\author{\name{Weihao Li$^{a}$, Dianne Cook$^{a}$, Emi Tanaka$^{a, b,
c}$, Susan VanderPlas$^{d}$}
\affil{$^{a}$Department of Econometrics and Business Statistics, Monash
University, Clayton, VIC, Australia; $^{b}$Biological Data Science
Institute, Australian National University, Acton, ACT,
Australia; $^{c}$Research School of Finance, Actuarial Studies and
Statistics, Australian National University, Acton, ACT,
Australia; $^{d}$Department of Statistics, University of Nebraska,
Lincoln, Nebraska, USA}
}

\thanks{CONTACT Weihao
Li. Email: \href{mailto:weihao.li@monash.edu}{\nolinkurl{weihao.li@monash.edu}}, Dianne
Cook. Email: \href{mailto:dicook@monash.edu}{\nolinkurl{dicook@monash.edu}}, Emi
Tanaka. Email: \href{mailto:emi.tanaka@anu.edu.au}{\nolinkurl{emi.tanaka@anu.edu.au}}, Susan
VanderPlas. Email: \href{mailto:susan.vanderplas@unl.edu}{\nolinkurl{susan.vanderplas@unl.edu}}}

\maketitle

\begin{abstract}
Regression experts consistently recommend plotting residuals for model
diagnosis, despite the existence of numerous hypothesis test procedures.
This paper provides evidence for why this is good advice, using data
from a visual inference experiment. We show how conventional tests are
too sensitive to be practically useful, and how the lineup protocol can
be used to yield reliable and consistent reading of residual plots for
better model diagnosis.
\end{abstract}

\begin{keywords}
statistical graphics; data visualization; visual inference; hypothesis
testing; reression analysis; cognitive perception; simulation; practical
significance; effect size
\end{keywords}

\hypertarget{introduction}{%
\section{Introduction}\label{introduction}}

\begin{quote}
\emph{``Since all models are wrong the scientist must be alert to what
is importantly wrong.''} \citep{box1976science}
\end{quote}

Diagnosing a model is an important part of analysing data. In linear
regression analysis, studying the residuals from a model fit is a common
diagnostic activity. Residuals summarise what is not captured by the
model, and thus provide the capacity to identify what might be wrong.

We can assess residuals in multiple ways. To examine the their
univariate distribution, residuals may be plotted as a histogram or
normal probability plot. Using the classical normal linear regression
model as an example, if the distribution is symmetric and unimodal, we
would consider it to be well-behaved. However, if the distribution is
skewed, bimodal, multimodal, or contains outliers, there would be cause
for concern. We can also inspect the distribution by conducting a
goodness-of-fit test, such as the Shapiro-Wilk normality test
\citep{shapiro1965analysis}.

In addition, scatterplots of residuals plotted against the fitted values
and each of the explanatory variables, can be used to scrutinize their
relationships. If there are any visually discoverable patterns, the
model is potentially inadequate or incorrectly specified. In general,
one looks for noticeable departures from the model such as non-linear
pattern or heteroskedasticity. A non-linear pattern would suggest that
the model needs to have some additional non-linear terms.
Heteroskedasticity suggests that the error is dependent on the
predictors, and hence violates the independence assumption. However,
correctly judging whether NO pattern exists in a residual plot is a
difficult task for humans. Humans will almost always will see a pattern
(see \citet{kahneman}), so the question that really needs answering is
whether any pattern perceived is consistent with randomness, purely
sampling variability, or noise. It is especially difficult to teach this
to new analysts and students \citep{loy2021bringing}. To address this,
statistical tests are available to check for non-linear patterns
\citep[e.g.][]{ramsey1969tests}, and heteroskedasticity
\citep[e.g.][]{breusch1979simple}.

Linear regression is a well-established procedure, and there is
considerable literature describing diagnostic procedures, e.g.
\citet{draper1998applied}, \citet{montgomery1982introduction},
\citet{belsley1980regression}, \citet{cook1999applied} and
\citet{cook1982residuals}. Interestingly, despite the abundance of
conventional tests, ALL of these writings advise that plotting residuals
is an essential tool for diagnosing regression model problems:

\begin{quote}
\emph{Some most useful checks allowed by data plots should be done on a
routine basis for every regression.} \citep{draper1998applied}
\end{quote}

\begin{quote}
\emph{Such formal and informal procedures are complementary, and both
have a place in residual analysis.} \citep{cook1982residuals}
\end{quote}

\begin{quote}
\emph{In our experience, statistical tests on regression model residuals
are not widely used. In most practical situations the residual plots are
more informative than the corresponding tests. However, since residual
plots do require skill and experience to interpret, the statistical
tests may occasionally prove useful.} \citep{montgomery1982introduction}
\end{quote}

\noindent The common wisdom of experts is that plotting the residuals is
indispensable for diagnosing model fits. The ubiquity of this advice is
\emph{curious}: investigating why, is the subject of this paper.

The paper is structured as follows. The next section describes the
background on the types of departures that one expects to detect, and
outlines a formal statistical process for reading residual plots, called
visual inference. Section \ref{experimental-design} details the
experimental design to compare the decisions made by formal hypothesis
testing, and how humans would read diagnostic plots. The results are
reported in Section \ref{results}. We conclude with a discussion of the
presented work, and ideas for future directions.

\hypertarget{background}{%
\section{Background}\label{background}}

\hypertarget{departures-from-good-residual-plots}{%
\subsection{Departures from good residual
plots}\label{departures-from-good-residual-plots}}

Graphical summaries where residuals are plotted against fitted values,
or other functions of the predictors (expected to be approximately
orthogonal to the residuals) are considered to be the most important
residual plots by \citet{cook1999applied}. Figure
\ref{fig:residual-plot-common-departures}A shows an example of an ideal
residual plot where points are symmetrically distributed around the
horizontal zero line (red), with no discernible patterns. There can be
various types of departures from this ideal pattern. Non-linearity,
heteroskedasticity and non-normality, shown in Figure
\ref{fig:residual-plot-common-departures}B,C,D are three commonly
checked departures.

\begin{figure}[t!]

{\centering \includegraphics[width=1\linewidth]{paper_files/figure-latex/residual-plot-common-departures-1} 

}

\caption{Example residual vs fitted value plots (red line indicates 0): (A) classically good looking residuals, (B) non-linear pattern indicates that the model has not captured a non-linear association, (C) heteroskedasticity indicating that variance around the fitted model is not uniform, and (D) non-normality where the residual distribution is not symmetric around 0. The latter pattern might best be assessed using a univariate plot of the residuals, but patterns B and C need to be assessed using a residual vs fitted value plot.}\label{fig:residual-plot-common-departures}
\end{figure}

Model misspecification occurs if functions of predictors that needed to
accurately describe the relationship between response and predictors are
omitted. Any non-linear pattern visible in the residual plot could be
indicative of this problem. An example residual plot containing visual
pattern of non-linearity is shown in Figure
\ref{fig:residual-plot-common-departures}B. One can clearly observe the
``S-shape'' from the residual plot, which corresponds to cubic term that
should have been included in the model.

Heteroskedasticity refers to the presence of non-constant error variance
in a regression model. It indicates that the distribution of residuals
depends on the predictors, violating the independence assumption. This
can be seen in a residual plot as an inconsistent spread of the
residuals relative to the fitted values or predictors. An example is the
``butterfly'' shape shown in Figure
\ref{fig:residual-plot-common-departures}C, or a ``left-triangle'' and
``right-triangle'' shape where the smallest variance occurs at one side
of the horizontal axis.

Figure \ref{fig:residual-plot-common-departures}D shows a scatterplot
where the residuals have a skewed distribution, as seen by the uneven
vertical spread. Unlike non-linearity and heteroskedasticity,
non-normality is usually detected with a different type of residual
plot: a histogram or normal probability plot. Because we focus on
scatterplots, non-normality is not one of the departures examined in
depth in this paper. (\citet{loy2016variations} discuss related work on
non-normality checking.)

\hypertarget{conventionally-testing-for-departures}{%
\subsection{Conventionally testing for
departures}\label{conventionally-testing-for-departures}}

Many different hypothesis tests are available to detect specific model
defects. For example, the presence of heteroskedasticity can usually be
tested by applying the White test \citep{white1980heteroskedasticity} or
the Breusch-Pagan (BP) test \citep{breusch1979simple}, which are both
derived from the Lagrange multiplier test \citep{silvey1959lagrangian}
principle that relies on the asymptotic properties of the null
distribution. To test specific forms of non-linearity, one may apply the
F-test as a model structural test to examine the significance of
specific polynomial and non-linear forms of the predictors, or the
significance of proxy variables as in the Ramsey Regression Equation
Specification Error Test (RESET) \citep{ramsey1969tests}. The
Shapiro-Wilk (SW) normality test \citep{shapiro1965analysis} is the most
widely used test of non-normality included by many of the statistical
software programs. The Jarque-Bera test \citep{jarque1980efficient} is
also used to directly check whether the sample skewness and kurtosis
match a normal distribution.

Table \ref{tab:example-residual-plot-table} displays the \(p\)-values
from the RESET, BP and SW tests applied to the residual plots in Figure
\ref{fig:residual-plot-common-departures}. The RESET test and BP test
were computed using the \texttt{resettest} and \texttt{bptest} functions
from the R package \texttt{lmtest}, respectively. The SW test was
computed using the \texttt{shapiro.test} from the core R package
\texttt{stats}. \footnote{Although we didn't use it, it's useful to know
  that the R package \texttt{skedastic} \citep{skedastic} also contains
  a large collection of functions to test for heteroskedasticity.}
Although, the RESET test is exact, it requires the selection of a power
parameter. \citet{ramsey1969tests} recommends a power of four, which we
adopted in our analysis. The BP and SW tests are approximate.

We would expect the RESET test for non-linearity to reject residual plot
B, the BP test for heteroskedasticity to reject the residual plot C, and
the SW test for non-normality to reject residual plot D, which they all
do and all tests also correctly fail to reject residual plot A.
Interestingly, the BP and SW tests also reject the residual plots
exhibiting structure that they weren't designed for.
\citet{cook1982residuals} explain that most residual-based tests for
particular types of departure from model assumptions are also sensitive
to other types of departures. This could be considered a Type III error
\citep{kimball1957errors}, where the null hypothesis of good residuals
is correctly rejected but for the wrong reason. Also, some types of
departure can have elements of other types of departure, for example,
non-linearity could be viewed as heteroskedasticity. Additionally, other
data problems such as outliers can trigger rejection (or not) of the
null hypothesis \citep{cook1999applied}.

With large sample sizes, hypothesis tests may reject the null hypothesis
when there is only a small effect. (A good discussion can be found in
\citet{kirk1996}.) While such rejections may be statistically correct,
their sensitivity may render the results impractical. A key goal of
residual plot diagnostics is to identify potential issues that could
lead to incorrect conclusions or errors in subsequent analyses, but
minor defects in the model are unlikely to have a significant impact and
may be best disregarded for practical purposes. The experiment discussed
in this paper specifically addresses this tension between statistical
and practical statistics.

\begin{table}

\caption{\label{tab:example-residual-plot-table}Statistical significance testing for departures from good residuals for plots in Figure \ref{fig:residual-plot-common-departures}. Shown are the $p$-values calculated for the RESET, the BP and the SW tests. The good residual plot (A) is judged a good residual plot, as expected, by all tests. The non-linearity (B) is detected by all tests, as might be expected given the extreme structure.}
\centering
\begin{tabular}[t]{llrrr}
\toprule
Plot & Departures & RESET & BP & SW\\
\midrule
A & None & 0.779 & 0.133 & 0.728\\
B & Non-linearity & \em{0.000} & \em{0.000} & \em{0.039}\\
C & Heteroskedasticity & 0.658 & \em{0.000} & \em{0.000}\\
D & Non-normality & 0.863 & 0.736 & \em{0.000}\\
\bottomrule
\end{tabular}
\end{table}

\hypertarget{visual-test-procedure-based-on-lineups}{%
\subsection{Visual test procedure based on
lineups}\label{visual-test-procedure-based-on-lineups}}

The examination of data plots to infer signal or patterns (or lack
thereof) is fraught with variation in the human ability to interpret and
decode the information embedded in a graph
\citep{cleveland1984graphical}. Human examination of diagnostic plots
can feel subjective. Ideally we would expect visual evaluation to be
done objectively with small amounts of human-to-human variation in
reading data plots, but we cannot assume this is the case.

In practice, over-interpretation of a single plot is common. For
instance, \citet{roy2015using} described a published example where
authors over-interpreted separation between gene groups from a
two-dimensional projection of a linear discriminant analysis.
\citet{roy2015using} showed that there were no differences in the
expression levels between the gene groups. One solution to
over-interpretation is to examine the plot in the context of natural
sampling variability assumed by the model, called the lineup protocol,
as proposed in \citet{buja2009statistical}.
\citet{majumder2013validation} showed that the lineup protocol is
analogous to the null hypothesis significance testing framework. The
protocol consists of \(m\) randomly placed plots, where one plot is the
data plot, and the remaining \(m - 1\) plots, referred to as the
\emph{null plots}, are constructed using the same graphical procedure as
the data plot but the data is replaced with null data that is generated
in a manner consistent with the null hypothesis, \(H_0\). Then, an
observer who has not seen the data plot is asked to point out the most
different plot from the lineup. Under \(H_0\), it is expected that the
data plot would have no distinguishable difference from the null plots,
and the probability that the observer correctly picks the data plot is
\(1/m\). If one rejects \(H_0\) as the observer correctly picks the data
plot, then the Type I error of this test is \(1/m\). This protocol
requires a priori specification of \(H_0\) (or at least a null data
generating mechanism), much like the requirement of knowing the sampling
distribution of the test statistic in null hypothesis significance
testing framework.

\begin{figure}[t!]

{\centering \includegraphics[width=1\linewidth]{paper_files/figure-latex/first-example-lineup-1} 

}

\caption{Visual testing is conducted using a lineup, as in the example here. The residual plot computed from the observed data (plot $2^2 + 2$, exhibiting non-linearity) is embedded among 19 null plots, where the residuals are simulated from a standard error model. Computing the $p$-value requires that the lineup be examined by a number of human judges, each asked to select the most different plot. A small $p$-value would result from a substantial number selecting plot $2^2 + 2$.}\label{fig:first-example-lineup}
\end{figure}

Figure \ref{fig:first-example-lineup} is an example of a lineup
protocol. If the data plot at position \(2^2 + 2\) is identifiable, then
it is evidence for the rejection of \(H_0\). In fact, the actual
residual plot is obtained from a misspecified regression model with
missing non-linear terms.

Data used in the \(m - 1\) null plots needs to be simulated. In
regression diagnostics, sampling data consistent with \(H_0\) is
equivalent to sampling data from the assumed model. As
\citet{buja2009statistical} suggested, \(H_0\) is usually a composite
hypothesis controlled by nuisance parameters. Since regression models
can have various forms, there is no general solution to this problem,
but it sometimes can be reduced to a so called ``reference
distribution'' by applying one of the three methods: (i) sampling from a
conditional distribution given a minimal sufficient statistic under
\(H_0\), (ii) parametric bootstrap sampling with nuisance parameters
estimated under \(H_0\), and (iii) Bayesian posterior predictive
sampling. The conditional distribution given a minimal sufficient
statistic is the best justified reference distribution among the three
\citep{buja2009statistical}. Essentially, null residuals can be
simulated by regressing \(N\) i.i.d standard normal random draws on the
predictors, then re-scaling it by the ratio of residual sum of square in
two regressions.

The effectiveness of lineup protocol for regression analysis has been
validated by \citet{majumder2013validation} under relatively simple
settings with up to two predictors. Their results suggest that visual
tests are capable of testing the significance of a single predictor with
a similar power to a t-test, though they express that in general it is
unnecessary to use visual inference if there exists a conventional test,
and they do not expect the visual test to perform equally well as the
conventional test. In their third experiment, where there is not a
conventional test, visual test outperforms the conventional test by a
large margin. This is encouraging, as it supports the use of visual
inference in situations where there are no existing statistical testing
procedures. Visual inference has also been integrated into diagnostics
for hierarchical linear models where the lineup protocol is used to
judge the assumptions of linearity, normality and constant error
variance for both the level-1 and level-2 residuals
\citep[\citet{loy2014hlmdiag} and
\citet{loy2015you}]{loy2013diagnostic}.

\hypertarget{calculation-of-statistical-significance-and-test-power}{%
\section{Calculation of statistical significance and test
power}\label{calculation-of-statistical-significance-and-test-power}}

\hypertarget{what-is-being-tested}{%
\subsection{What is being tested?}\label{what-is-being-tested}}

In diagnosing a model fit using the residuals, we are generally
interested in testing whether ``\emph{the regression model is correctly
specified}'' (\(H_0\)) against the broad alternative ``\emph{the
regression model is misspecified}'' (\(H_a\)).

However, it is practically impossible to test this broad \(H_0\) with
conventional tests, because they need specific structure causing the
departure to be quantifiable in order to be computable. For example, the
RESET test for detecting non-linear departures is formulated by fitting
\(y = \tau_0 + \sum_{i=1}^{p}\tau_px_p +\gamma_1\hat{y}^2 + \gamma_2\hat{y}^3 + \gamma_3\hat{y}^4 + u, ~~u \sim N(0, \sigma_u^2)\)
in order to test \(H_0:\gamma_1 = \gamma_2 = \gamma_3 = 0\) against
\(H_a: \gamma_1 \neq 0 \text{ or } \gamma_2 \neq 0 \text{ or } \gamma_3 \neq 0\).
Similarly, the BP test is designed to specifically test \(H_0:\)
\emph{error variances are all equal}
(\(\zeta_i=0 \text{ for } i=1,..,p\)) versus the alternative \(H_a:\)
\emph{that the error variances are a multiplicative function of one or
more variables} (\(\text{at least one } \zeta_i\neq 0\)) from
\(e^2 = \zeta_0 + \sum_{i=1}^{p}\zeta_i x_i + u, ~ u\sim N(0,\sigma_u^2)\).

While a battery of conventional tests for different types of departures
could be applied, this is intrinsic to the lineup protocol. The lineup
protocol operates as an omnibus test, able to detect a range of
departures from good residuals in a single application.

\hypertarget{statistical-significance}{%
\subsection{\texorpdfstring{Statistical
significance\label{sig}}{Statistical significance}}\label{statistical-significance}}

In hypothesis testing, a \(p\)-value is defined as the probability of
observing test results at least as extreme as the observed result
assuming \(H_0\) is true. Conventional hypothesis tests usually have an
existing method to derive or compute the \(p\)-value based on the null
distribution. Here we establish the method to estimate a \(p\)-value for
a visual test.

Within the context of visual inference, with \(K\) independent
observers, the visual \(p\)-value can be seen as the probability of
having as many or more participants detect the data plot than the
observed result.

The approach used in \citet{majumder2013validation} is as follows.
Define \(X_j = \{0,1\}\) to be a Bernoulli random variable measuring
whether participant \(j\) detected the data plot, and
\(X = \sum_{j=1}^{K}X_j\) be the total number of observers who detected
the data plot. Then, by imposing a relatively strong assumption that all
\(K\) evaluations are fully independent, under \(H_0\)
\(X \sim \mathrm{Binom}_{K,1/m}\). Therefore, the \(p\)-value of a
lineup of size \(m\) evaluated by \(K\) observer is estimated with
\(P(X \geq x) = 1 - F(x) + f(x)\), where \(F(.)\) is the binomial
cumulative distribution function, \(f(.)\) is the binomial probability
mass function and \(x\) is the realization of number of observers
choosing the data plot.

As pointed out by \citet{vanderplas2021statistical}, this basic binomial
model is deficient. It does not take into account the possible
dependencies in the visual test due to repeated evaluations of the same
lineup, or account for when participants are offered the option to
select one or more ``most different'' plots, or none, from a lineup.
They suggest three common lineup scenarios: (1) \(K\) different lineups
are shown to \(K\) participants, (2) \(K\) lineups with different null
plots but the same data plot are shown to \(K\) participants, and (3)
the same lineup is shown to \(K\) participants. Scenario 3 is the most
feasible to apply, but has the most dependencies to accommodate for the
\(p\)-value calculation. For Scenario 3, VanderPlas et al propose
modeling the probability of plot \(i\) being selected from a lineup as
\(\theta_i\), where \(\theta_i \sim Dirichlet(\alpha)\) for
\(i=1,...,m\) and \(\alpha > 0\). The number of times plot \(i\) being
selected in \(K\) evaluations is denoted as \(c_i\). In case participant
\(j\) makes multiple selections, \(1/s_j\) will be added to \(c_i\)
instead of one, where \(s_j\) is the number of plots participant \(j\)
selected for \(j=1,...K\). This ensures \(\sum_{i}c_i=K\). Since we are
only interested in the selections of the data plot \(i\), the marginal
model can be simplified to a beta-binomial model and thus the visual
\(p\)-value is given as

\begin{equation} \label{eq:pvalue-beta-binomial}
P(C \geq c_i) = \sum_{x=c_i}^{K}{K \choose x}\frac{B(x + \alpha, K - x + (m - 1)\alpha)}{B(\alpha, (m-1)\alpha)},\quad \text{for} \quad c_i \in \mathbb{Z}_0^+
\end{equation}

\noindent where \(B(.)\) is the beta function defined as

\begin{equation} \label{eq:betafunction}
B(a, b) = \int_{0}^{1}t^{\alpha - 1}(1-t)^{b-1}dt,\quad \text{where}\quad a,b>0.
\end{equation}

\noindent We extend the equation to non-negative real number \(c_i\) by
applying a linear approximation

\begin{equation} \label{eq:pvalue-beta-binomial-approx}
P(C \geq c_i) = P(C \geq \lceil c_i \rceil) + (\lceil c_i \rceil - c_i) P(C = \lfloor c_i \rfloor), \quad \text{for}\quad c_i \in \mathbb{R}_0^+,
\end{equation}

\noindent where \(P(C \geq \lceil c_i \rceil)\) is calculated using
Equation \ref{eq:pvalue-beta-binomial} and
\(P(C = \lfloor c_i \rfloor)\) is calculated by

\begin{equation} \label{eq:pmf-beta-binomial}
P(C = c_i) = {K \choose c_i}\frac{B(c_i + \alpha, K - c_i + (m - 1)\alpha)}{B(\alpha, (m-1)\alpha)},\quad \text{for} \quad c_i \in \mathbb{Z}_0^+.
\end{equation}

The parameter \(\alpha\) used in Equation \ref{eq:pvalue-beta-binomial}
and \ref{eq:pmf-beta-binomial} is usually unknown and will need to be
estimated from the survey data. An interpretation of \(\alpha\) is that
when it is low only a few plots are attractive to the observers and tend
to be selected, and when high, most plots are equally likely to be
chosen. VanderPlas et al define \(c\)-interesting plot to be if \(c\) or
more participants select the plot as the most different. The expected
number of plots selected at least \(c\) times, \(E[Z_c]\), is then
calculated as

\begin{equation} \label{eq:c-interesting-expectation}
E[Z_c(\alpha)] = \frac{m}{B(\alpha, (m-1)\alpha)}\sum_{\lceil c \rceil}^{K}{K \choose x} B(x + \alpha, K - x + (m-1)\alpha).\end{equation}

With Equation \ref{eq:c-interesting-expectation}, \(\alpha\) can be
estimated using maximum likelihood estimation. Precise estimation of
\(\alpha\), is aided by evaluation of Rorschach lineups, where all plots
are null plots. In a Rorschach, in theory all plots should be equally
likely, but in practice some (irrelevant) visual elements may be more
eye-catching than others. This is what \(alpha\) captures, the capacity
for extraneous features to distract the observer for a particular type
of plot display.

\hypertarget{power-of-the-tests}{%
\subsection{Power of the tests}\label{power-of-the-tests}}

The power of a model misspecification test is the probability that
\(H_0\) is rejected given the regression model is misspecified in a
specific way. It is an important indicator when one is concerned about
whether model assumptions have been violated. In practice, one might be
more interested in knowing how much the residuals deviate from the model
assumptions, and whether this deviation is of practical significance.

The power of a conventional hypothesis test is affected by both the true
parameter \(\boldsymbol{\theta}\) and the sample size \(n\). These two
can be quantified in terms of effect size \(E\) to measure the strength
of the residual departures from the model assumptions. Details about the
calculation of effect size are provided in Section \ref{effect-size}
after the introduction of the simulation model used in our experiment.
The theoretical power of a test is sometimes not a trival solution, but
it can be estimated if the data generating process is known. We use a
predefined model to generate a large set of simulated data under
different effect sizes, and record if the conventional test rejects
\(H_0\). The probability of the conventional test rejects \(H_0\) is
then fitted by a logistic regression formulated as

\begin{equation} \label{eq:logistic-regression-1-1}
Pr(\text{reject}~H_0|H_1,E) = \Lambda\left(log\left(\frac{0.05}{0.95}\right) + \beta_1 E\right),
\end{equation}

\noindent where \(\Lambda(.)\) is the standard logistic function given
as \(\Lambda(z) = exp(z)/(1+exp(z))\). The effect size \(E\) is the only
predictor and the intercept is fixed to \(log(0.05/0.95)\) so that
\(\hat{Pr}(\text{reject}~H_0|H_1,E = 0) = 0.05\), the desired
significance level.

The power of a visual test on the other hand, may additionally depend on
the ability of the particular participant, as the skill of each
individual may affect the number of observers who identify the data plot
from the lineup \citep{majumder2013validation}. To address this issue,
\citet{majumder2013validation} models the probability of participant
\(j\) correctly picking the data plot from lineup \(l\) using a
mixed-effect logistic regression, with participants treated as random
effect. Then, the estimated power of a visual test evaluated by a single
participant is the predicted value obtained from the mixed effects
model. However, this mixed effects model does not work with scenario
where participants are asked to select one or more most different plots.
In this scenario, having the probability of a participant \(j\)
correctly picking the data plot from a lineup \(l\) is insufficient to
determine the power of a visual test because it does not provide
information about the number of selections made by the participant for
the calculation of the \(p\)-value (See Equation
\ref{eq:pvalue-beta-binomial-approx}). Therefore, we directly estimate
the probability of a lineup being rejected by assuming that individual
skill has negligible effect on the variation of the power. This
assumption essentially averages out the subject ability and helps to
simplify the model structure, thereby obviating a costly large-scale
experiment to estimate complex covariance matrices. The same model given
in Equation \ref{eq:logistic-regression-1-1} is applied to model the
power of a visual test.

To study various factors contributing to the power of both tests, the
same logistic regression model is fit on different subsets of the
collated data grouped by levels of factors. These include the
distribution of the fitted values, type of the simulation model and the
shape of the residual departures.

\hypertarget{experimental-design}{%
\section{Experimental design}\label{experimental-design}}

Our experiment was conducted over three data collection periods to
investigate the difference between conventional hypothesis testing and
visual inference in the application of linear regression diagnostics.
Two types of departures, non-linearity and heteroskedasticity, were
collected during data collection periods I and II. The data collection
period III was designed primarily to measure human responses to null
lineups so that the parameter \(\alpha\) in Equation
\ref{eq:pvalue-beta-binomial} can be estimated. Additional lineups for
both non-linearity and heteroskedasticity, using uniform fitted value
distributions, were included for additional data, and to avoid
participant frustration of too many difficult tasks. Overall, we
collected 7974 evaluations on 1152 unique lineups performed by 443
participants throughout three data collection periods.

A summary of the factors used in the experiment can be found in Table
\ref{tab:model-factor-table}. There were four levels of the non-linear
structure, and three levels of heteroskedastic structure. The signal
strength was controlled by error variance (\(\sigma\)) for the
non-linear pattern, and by a ratio (\(b\)) parameter for the
heteroskedasticity. Additionally, three levels of sample size (\(n\))
and four different fitted value distributions were incorporated.

\hypertarget{simulating-departures-from-good-residuals}{%
\subsection{Simulating departures from good
residuals}\label{simulating-departures-from-good-residuals}}

\hypertarget{non-linearity}{%
\subsubsection{Non-linearity}\label{non-linearity}}

\begin{table}

\caption{\label{tab:model-factor-table}Levels of the factors used in data collection periods I, II, III.}
\centering
\resizebox{\linewidth}{!}{
\begin{tabular}[t]{rr|rr|rcrr|rr|rcrr|rr|rcrr|rr|rcrr|rr|rcrr|rr|rc}
\toprule
\multicolumn{2}{c}{Non-linearity} & \multicolumn{2}{c}{Heteroskedasticity} & \multicolumn{2}{c}{Common} \\
\cmidrule(l{3pt}r{3pt}){1-2} \cmidrule(l{3pt}r{3pt}){3-4} \cmidrule(l{3pt}r{3pt}){5-6}
\multicolumn{1}{c}{Poly Order ($j$)} & \multicolumn{1}{c}{SD ($\sigma$)} & \multicolumn{1}{c}{Shape ($a$)} & \multicolumn{1}{c}{Ratio ($b$)} & \multicolumn{1}{c}{Size ($n$)} & \multicolumn{1}{c}{Distribution of fitted values} \\
\cmidrule(l{3pt}r{3pt}){1-1} \cmidrule(l{3pt}r{3pt}){2-2} \cmidrule(l{3pt}r{3pt}){3-3} \cmidrule(l{3pt}r{3pt}){4-4} \cmidrule(l{3pt}r{3pt}){5-5} \cmidrule(l{3pt}r{3pt}){6-6}
2 & 0.25 & -1 & 0.25 & 50 & Uniform\\
3 & 1.00 & 0 & 1.00 & 100 & Normal\\
6 & 2.00 & 1 & 4.00 & 300 & Skewed\\
18 & 4.00 &  & 16.00 &  & Discrete\\
 &  &  & 64.00 &  & \\
\bottomrule
\end{tabular}}
\end{table}

Data collection period I was designed to study the ability of
participants to detect the effect of a non-linear term
\(\boldsymbol{z}\) constructed using Hermite polynomials on random
vector \(\boldsymbol{x}\) formulated as

\begin{align*} \label{eq:nonlinearity-model}
\boldsymbol{y} &= 1 + \boldsymbol{x} + \boldsymbol{z} + \boldsymbol{\varepsilon},\\
\boldsymbol{x} &= g(\boldsymbol{x}_{raw}, 1), \\
\boldsymbol{z} &= g(\boldsymbol{z}_{raw}, 1), \\
\boldsymbol{z}_{raw} &= He_j(g(\boldsymbol{x}, 2)),
\end{align*}

\noindent where \(\boldsymbol{y}\), \(\boldsymbol{x}\),
\(\boldsymbol{\varepsilon}\), \(\boldsymbol{x}_{raw}\),
\(\boldsymbol{z}_{raw}\) are vectors of size \(n\), \(He_{j}(.)\) is the
\(j\)th-order probabilist's Hermite polynomials,
\(\varepsilon \sim N(\boldsymbol{0}, \sigma^2\boldsymbol{I}_n)\), and
\(g(\boldsymbol{x}, k)\) is a scaling function to enforce the support of
the random vector to be \([-k, k]^n\) defined as

\begin{equation} \label{eq:scaling-function}
g(\boldsymbol{x}, k) = (\boldsymbol{x} - min(\boldsymbol{x}))/max(\boldsymbol{x} - min(\boldsymbol{x}))2k - k, \quad \text{for} \quad k > 0. 
\end{equation}

According to \citet{abramowitz1964handbook}, Hermite polynomials were
initially defined by \citet{de1820theorie}, but named after
\citet{hermite1864nouveau} because of the unrecognisable form of
Laplace's work. The function \texttt{hermite} from the R package
\texttt{mpoly} \citep{mpoly} is used to simulate
\(\boldsymbol{z}_{raw}\) to generate Hermite polynomials.

The null regression model used to fit the realizations generated by the
above model is formulated as

\begin{equation} \label{eq:null-model}
\boldsymbol{y} = \beta_0 + \beta_1 \boldsymbol{x} + \boldsymbol{u},
\end{equation}

\noindent where
\(\boldsymbol{u} \sim N(\boldsymbol{0}, \sigma^2\boldsymbol{I}_n)\).
Since \(z = O(x^j)\), for \(j > 1\), \(z\) is a higher order term left
out of the null regression, which will lead to model misspecification.

\begin{figure}[!h]

{\centering \includegraphics[width=1\linewidth]{paper_files/figure-latex/different-shape-of-herimite-1} 

}

\caption{Polynomial forms generated for the residual plots used to assess detecting non-linearity. The four shapes are generated by varying the order of polynomial given by $j$ in $He_j(.)$.}\label{fig:different-shape-of-herimite}
\end{figure}

\begin{figure}[!h]

{\centering \includegraphics[width=1\linewidth]{paper_files/figure-latex/different-sigma-1} 

}

\caption{Examining the effect of $\sigma$ on the signal strength in the non-linearity detection, for $n=300$, uniform fitted value distribution and the "U" shape. As $\sigma$ increases the signal strength decreases, to the point that the "U" is almost unrecognisable when $\sigma=4$.}\label{fig:different-sigma}
\end{figure}

Visual patterns of non-linearity are simulated using four different
orders of probabilist's Hermite polynomials (\(j = 2, 3, 6, 18\)). The
values of \(j\) were chosen so that distinct shapes of non-linearity
were included in the residual plot. These include ``U'', ``S'', ``M''
and ``triple-U'' shape as shown in Figure
\ref{fig:different-shape-of-herimite}. A greater value of \(j\) will
result in a curve with more turning points. It is expected that the
``U'' shape will be the easiest detect, as the shape gets more complex
it will be harder to perceive in a scatterplot, particularly when there
is noise. Figure \ref{fig:different-sigma} shows the ``U'' shape for
different amounts of noise (\(\sigma\)).

Figure \ref{fig:example-poly-lineup} is one of the non-linearity lineups
used in experiment. This data plot is produced by the non-linearity
model with \(j = 6\), and is in location \(2^3+1\). All five
participants who viewed the lineup chose this plot as most different,
thus all detected the data plot.

\begin{figure}[t!]

{\centering \includegraphics[width=1\linewidth]{paper_files/figure-latex/example-poly-lineup-1} 

}

\caption{One of the lineups containing non-linearity patterns used in data collection period I. Can you spot the most different plot? The data plot is positioned at $2^3 + 1$.}\label{fig:example-poly-lineup}
\end{figure}

\hypertarget{heteroskedasticity}{%
\subsubsection{Heteroskedasticity}\label{heteroskedasticity}}

Data collection period II was designed to study the ability of
participants to detect the heteroskedasticity under a simple linear
regression model setting:

\begin{align*} \label{eq:heter-model}
\boldsymbol{y} &= 1 + \boldsymbol{x} + \boldsymbol{\varepsilon},\\
\boldsymbol{x} &= g(\boldsymbol{x}_{raw}, 1),\\
\boldsymbol{\varepsilon} &\sim N(\boldsymbol{0}, 1 + (2 - |a|)(\boldsymbol{x} - a)^2b \boldsymbol{I}), 
\end{align*}

\noindent where \(\boldsymbol{y}\), \(\boldsymbol{x}\),
\(\boldsymbol{\varepsilon}\) are vectors of size \(n\) and \(g(.)\) is
the scaling function defined in Equation \ref{eq:scaling-function}. The
null regression model used to fit the realizations generated by the
above model is formulated exactly the same as Equation
\ref{eq:null-model}.

For \(b \neq 0\), the variance-covariance matrix of the error term
\(\boldsymbol{\varepsilon}\) is correlated with the predictor
\(\boldsymbol{x}\), which will lead to the presence of
heteroskedasticity. Visual patterns of heteroskedasticity are simulated
using three different shapes (\(a\) = -1, 0, 1).

Since \(supp(X) = [-1, 1]\), choosing \(a\) to be \(-1\), \(0\) and
\(1\) can generate ``left-triangle'', ``butterfly'' and
``right-triangle'' shapes as displayed in Figure
\ref{fig:different-shape-of-heter}. The term \((2 - |a|)\) maintains the
magnitude of residuals across different values of \(a\). Figure
\ref{fig:different-b} shows the butterfly shape as the ratio parameter
(\(b\)) is changed.

\begin{figure}[!h]

{\centering \includegraphics[width=1\linewidth]{paper_files/figure-latex/different-shape-of-heter-1} 

}

\caption{Heteroskedasticity forms used in the experiment. Three different shapes ($a = -1, 0, 1$) are used in the experiment to create left-triangle, "butterfly" and "right-triangle" shapes, respectively.}\label{fig:different-shape-of-heter}
\end{figure}

\begin{figure}[!h]

{\centering \includegraphics[width=1\linewidth]{paper_files/figure-latex/different-b-1} 

}

\caption{Five different values of $b$ are used in heteroskedasticity simulation to control the strength of the signal. Larger values of $b$ yield a bigger difference in variation, and stus stronger heteroskedasticity signal.}\label{fig:different-b}
\end{figure}

\begin{figure}[t!]

{\centering \includegraphics[width=1\linewidth]{paper_files/figure-latex/example-heter-lineup-1} 

}

\caption{One of the lineups containing heteroskedasticity pattern used in data collection period II. Can you spot the most different plot? The data plot is positioned at $3^3 - 3^2$.}\label{fig:example-heter-lineup}
\end{figure}

An example lineup of this model used in data collection period II is
shown in Figure \ref{fig:example-heter-lineup} with \(a = -1\). The data
plot location is in position \(2^4 + 2\). Nine out of 11 participants
chose this plot from the lineup.

\hypertarget{factors-common-to-both-data-collection-periods}{%
\subsubsection{Factors common to both data collection
periods}\label{factors-common-to-both-data-collection-periods}}

Fitted values are a function of the independent variables, and the
distribution of the observed values affects the distribution of the
fitted values. In the best case scenario the fitted values will have a
uniform distribution, which means that there is even coverage of
possible observed values across all of the predictors. This is not
always present in the collected data. Sometimes the fitted values are
discrete because one or more predictors were measured discretely. The
distribution may be relatively Gaussian, reflecting a linear combination
of many predictors, adhering to the Central Limit Theorem. It is also
common to see a skewed distribution of fitted values if one or more of
the predictors has a skewed distribution. This latter problem is usually
corrected before modelling, using a variable transformation. Our
simulation assess this by using four different distributions to
represent fitted values, constructed by different sampling of the raw
predictor \(X_{raw}\):

\begin{itemize}
\tightlist
\item
  uniform, \(U(-1, 1)\),
\item
  normal, \(N(0, 0.3^2)\),
\item
  skewed, \(lognormal(0, 0.6^2)/3\), and
\item
  discrete, \(U\{1, 5\}\).
\end{itemize}

\noindent Figure \ref{fig:different-dist} shows the non-linear pattern,
a ``U'' shape, with the different fitted value distributions. We would
expect that structure in residual plots would be easier to perceive when
the fitted values are uniformly distributed.

Three different sample sizes were used in our experiment:
\(n = 50, 100, 300\). Figure \ref{fig:different-n} shows the non-linear
``S'' shape for different sample sizes. We expect signal strength to
decline in the simulated data plots with smaller \(n\). A sample size of
300 is typically enough for structure to be visible in a scatter plot
reliably. Beyond 300, structure can be obscured by over-plotting, and a
different type of display that included transparency, binning or density
mapping, might be needed.

\begin{figure}

{\centering \includegraphics[width=1\linewidth]{paper_files/figure-latex/different-dist-1} 

}

\caption{Variations in fitted values, that might affect perception of residual plots. Four different distributions are used.}\label{fig:different-dist}
\end{figure}

\begin{figure}

{\centering \includegraphics[width=1\linewidth]{paper_files/figure-latex/different-n-1} 

}

\caption{Examining the effect of signal strength for the three different values of $n$ used in the experiment, for non-linear structure with fixed $\sigma = 1.5$, uniform fitted value distribution, and "S" shape. For these factor levels, only when $n = 300$ is the "S" shape clearly visible.}\label{fig:different-n}
\end{figure}

\hypertarget{controlling-the-signal-strength}{%
\subsubsection{Controlling the signal
strength}\label{controlling-the-signal-strength}}

The three parameters \(n\), \(\sigma\) and \(b\) are important for
controlling the strength of the signal to generate lineups with a
variety difficulty levels. This will ensure that estimated power curves
will be smooth and continuous, and that participants are allocated a set
of lineups with a range of difficulty.

Parameter \(\sigma \in \{0.5, 1, 2, 4\}\) and
\(b \in \{0.25, 1, 4, 16, 64\}\) are used in data collection periods I
and II respectively. A large value of \(\sigma\) will increase the
variation of the error of the non-linearity model and decrease the
visibility of the visual pattern. The parameter \(b\) controls the
variation in the standard deviation of the error across the support of
the predictor. A larger value of \(b\) generates a larger ratio between
it's smallest and highest values, making the visual pattern more
obvious. The sample size \(n\) sharpens (or blurs) a pattern when it is
larger (or smaller). Figures \ref{fig:different-sigma},
\ref{fig:different-b} and \ref{fig:different-n} demonstrate the impact
of these parameters on signal strength.

\hypertarget{experimental-setup}{%
\subsection{Experimental setup}\label{experimental-setup}}

The lineups are allocated to participants in a manner that covers the
experimental design and is relatively uniform. To manage this, we use
effect size to measure the signal strength, which helps in assigning a
set of lineups with a range of difficulties to each participant.

\hypertarget{effect-size}{%
\subsubsection{Effect size}\label{effect-size}}

Effect size in statistics measures the strength of the signal relative
to the noise. It is surprisingly difficult to quantify, even for
simulated data as used in this experiment.

For the non-linearity model, the key items defining effect size are
sample size (\(n\)) and variance of the error term (\(\sigma^2\)), and
so effect size would be roughly calculated as \(\sqrt{n}/{\sigma}\). As
sample size increases the effect size would increase, but as variance
increases the effect size decreases. However, it is not clear how the
additional parameter for the model polynomial order, \(k\), should be
incorporated. Intuitively, the large \(k\) means more complex pattern,
which likely means effect size would decrease. For the purposes of our
calculations we have chosen to use an approach based on Kullback-Leibler
divergence \citep{kullback1951information}, coupled with simulation.
This formulation defines effect size to be:

\begin{equation}
E = \frac{1}{2}\left(\boldsymbol{\mu}_z'(diag(\boldsymbol{R}\sigma^2))^{-1}\boldsymbol{\mu}_z\right)
\label{eq:nonlin-effect-size}
\end{equation}

\noindent where \(diag(.)\) is the diagonal matrix constructed from the
diagonal elements of a matrix,
\(\boldsymbol{R} = \boldsymbol{I}_n - \boldsymbol{H}\) is the residual
operator,
\(\boldsymbol{H} = \boldsymbol{X}(\boldsymbol{X}'\boldsymbol{X})^{-1}\boldsymbol{X}'\)
is the hat matrix,
\(\boldsymbol{\mu}_z = \boldsymbol{R}\boldsymbol{Z}\boldsymbol{\beta}_z\)
is the expected values of residuals where \(\boldsymbol{Z}\) contain any
higher order terms of \(\boldsymbol{X}\) left out of the regression
equation, \(\boldsymbol{\beta}_z\) contains the corresponding
coefficients, and \(\sigma^2\boldsymbol{I}\) is the assumed covariance
matrix of the error term when \(H_0\) is true.

In the heteroskedasticity model, the key elements for measuring effect
size are sample size, \(n\), and the ratio of the biggest variance to
smallest variance, \(b\). Larger values of both would produce higher
effect size. However, it is not clear how to incorporate the additional
shape parameter, \(a\). Thus the same approach is used here, where the
formula can be written as:

\begin{equation}
E = \frac{1}{2}\left(log\frac{|diag(\boldsymbol{R}\boldsymbol{V}\boldsymbol{R}')|}{|diag(\boldsymbol{R})|} - n + tr(diag(\boldsymbol{R}\boldsymbol{V}\boldsymbol{R}')^{-1}diag(\boldsymbol{R}))\right)
\label{eq:hetero-effect-size}
\end{equation}

\noindent where \(\boldsymbol{V}\) is the actual covariance matrix of
the error term. Derivations for these equations are provided in the
Appendix.

To compute the effect size for each lineup we simulate a sufficient
large number of samples from the same model, in each sample, the number
of observations \(n\) is fixed. We then compute the effect size for each
sample and take the average as the final value. This ensures lineups
constructed with the same experimental factors will share the same
effect size.

\hypertarget{lineup-allocation-to-participants}{%
\subsubsection{Lineup allocation to
participants}\label{lineup-allocation-to-participants}}

From Table \ref{tab:model-factor-table} it can be computed that there
are a total of \(4 \times 4 \times 3 \times 4 = 192\) and
\(3 \times 5 \times 3 \times 4 = 180\) combinations of parameter values
for the non-linearity model and heteroskedasticity models respectively.
Three replications for each combination results in
\(192 \times 3 = 576\) and \(180 \times 3 = 540\) lineups, respectively.

Each lineup needs to be evaluated by at least five participants. From
previous work, and additional pilot studies for this experiment, we
decided to record evaluations from 20 lineups for each participant. Two
of the 20 lineups with clear visual patterns were used as attention
checks, to ensure quality data. Thus, \(576 \times 5 / (20-2) = 160\)
and \(540 \times 5 / (20-2) = 150\) participants were needed to cover
the experimental design for the data collection periods I and II,
respectively. The factor levels and range of difficulty was assigned
relatively equally among participants. The Appendix contains graphical
summaries of these assignments for each subject.

Data collection period III was primarily to obtain Rorschach lineup
evaluations to estimate \(\alpha\) from Equation
\ref{eq:pvalue-beta-binomial}, and also to obtain additional evaluations
of lineups made with uniform fitted value distribution to ensure
consistency in the results. To construct a Rorschach lineup, the data is
generated from a model with zero effect size, while the null data are
generated using the same simulation method discussed in Section
\ref{visual-test-procedure-based-on-lineups}. This procedure differs
from that of the canonical Rorschach lineup, where all 20 plots are
generated directly from the null model, so the method suggested in
\citet{vanderplas2021statistical} for typical lineups containing a data
plot is used to estimate \(\alpha\). (The Appendix contains a
sensitivity analysis on the effect of uncertainty in the estimation of
\(\alpha\).) The \(3 \times 4 = 12\) treatment levels of the common
factors, replicated three times results in 36 lineups. And 6 more
evaluations on the 279 lineups with uniform fitted value distribution,
results in at least
\((36 \times 20 + 279 \times 3 \times 6) / (20-2) = 133\) participants
needed.

\hypertarget{collecting-results}{%
\subsubsection{Collecting results}\label{collecting-results}}

Participants for all three data collection periods were recruited from
the Prolific crowd-sourcing platform \citep{palan2018prolific}.
Pre-screening procedures were applied during the recruitment:
participants were required to be fluent in English, with \(98\%\)
minimum approval rate and at least 10 submissions in other studies.

During the experiment, every participant is presented with a block of 20
lineups. A lineup consists of a randomly placed data plot and 19 null
plots, which are all residual plots drawn with raw residuals on the
y-axis and fitted values on the x-axis. An additional horizontal red
line is added at \(y = 0\) as a visual reference. The data in the data
plot is simulated from one of two models described in Section
\ref{simulating-departures-from-good-residuals}, while the data of the
remaining 19 null plots are generated by the residual rotation technique
discussed in Section \ref{visual-test-procedure-based-on-lineups}.

In each lineup evaluation, the participant was asked to select one or
more plots that are most different from others, provide a reason for
their selections, and evaluate how different they think the selected
plots are from others. If there is no noticeable difference between
plots in a lineup, participants had the option to select zero plots
without the need to provide a reason. During the process of recording
the responses, a zero selection was considered to be equivalent to
selecting all 20 plots. No participant was shown the same lineup twice.
Information about preferred pronouns, age group, education, and previous
experience in visual experiments were also collected. A participant's
submission was only included in the analysis if the data plot is
identified for at least one attention check.

\hypertarget{results}{%
\section{Results}\label{results}}

Data collection used a total of 1152 lineups, and resulted in a total of
7974 evaluations from 443 participants. Roughly half corresponded to the
two models, non-linearity and heteroskedasticiy, and the three
collection periods had similar numbers of evaluations. Each participant
received two of the 24 attention check lineups which were used to filter
results of participants who were clearly not making an honest effort
(only 11 of 454). To estimate \(\alpha\) for calculating statistical
significance (see Section \ref{sig}) there were 720 evaluations of 36
null lineups. Neither the attention checks nor null lineups were used in
the subsequent analysis. The de-identified data, \texttt{vi\_survey}, is
made available in the R package, \texttt{visage}.\footnote{The package
  \texttt{visag} is currently available on GitHub
  (\url{https://github.com/TengMCing/visage})}

The data was collected on lineups constructed from four different fitted
value distributions: uniform, normal, skewed and discrete. More data was
collected on the uniform distribution (each evaluated by 11
participants) than the others (each evaluated by 5 participants). The
analysis in Sections \ref{power-analysis}-\ref{hetero-analysis} uses
only results from lineups generated with uniform fitted values, for a
total 3069 lineup evaluations. This allows us to compare the
conventional and visual test performance in an optimal scenario. Section
\ref{effect-of-fitted-value-distributions} examines how the results may
be affected if the fitted value distribution was different.

\hypertarget{power-comparison-of-the-tests}{%
\subsection{\texorpdfstring{Power comparison of the
tests\label{power-analysis}}{Power comparison of the tests}}\label{power-comparison-of-the-tests}}

Figures \ref{fig:polypower} and \ref{fig:heterpower} present the power
curves of various tests plotted against the effect size in the residuals
for non-linearity and heteroskedasticity respectively. In each case the
power of visual test is calculated for multiple bootstrap samples
leading to the many (orange) curves. The effect size was computed at a
5\% significance level and plotted on a natural logarithmic scale. To
facilitate visual calibration of effect size values with the
corresponding diagnostic plots, a sequence of example residual plots
with increasing effect sizes is provided at the bottom of these figures.
These plots serve as a visual aid to help readers understand how
different effect size values translate to changes in the diagnostic
plots. The horizontal lines of dots at 0 and 1 represent the
non-rejection or rejection decisions made by visual tests for each
lineup.

\begin{figure}

{\centering \includegraphics[width=1\linewidth]{paper_files/figure-latex/polypower-1} 

}

\caption{Comparison of power between different tests for non-linear patterns (uniform fitted values only). The power curves are estimated using logistic regression, and the horizontal lines of dots represent non-reject and reject results from visual tests for each lineup. The visual test has multiple power curves estimated from bootstrap samples. The row of scatterplots at the bottom are examples of residual plots corresponding to the specific effect sizes marked by vertical lines in the main plot.}\label{fig:polypower}
\end{figure}

\begin{figure}

{\centering \includegraphics[width=1\linewidth]{paper_files/figure-latex/heterpower-1} 

}

\caption{Comparison of power between different tests for heteroskedasticity patterns (uniform fitted values only). Main plot shows the power curves, with dots indicating non-reject and reject in visual testing of lineups. The multiple lines for the visual test arise from estimating the power on many bootstrap samples. The row of scatterplots at the bottom are examples of residual plots corresponding to the specific effect sizes marked by vertical lines in the main plot.}\label{fig:heterpower}
\end{figure}

Figure \ref{fig:polypower} compares the power for the different tests
for non-linear structure in the residuals. The test with the uniformly
higher power is the RESET test, one that specifically tests for
non-linearity. Note that the BP and SW tests have much lower power,
which is expected because they are not designed to detect non-linearity.
The bootstrapped power curves for the visual test are effectively a
shift right from that of the RESET test. This means that the RESET test
will reject at a lower effect size (less structure) than the visual
test, but otherwise the performance will be similar. In other words, the
RESET test is more sensitive than the visual test. This is not
necessarily a good feature for the purposes of diagnosing model defects:
If we scan the residual plot examples at the bottom, we might argue that
the non-linearity is not sufficiently problematic until an effect size
of around 3 or 3.5. The RESET test would reject closer to an effect size
of 2, but the visual test would reject closer 3.25, for a significance
level of 0.05. The visual test matches the robustness of the model to
(minor) violations of assumptions much better.

For the heteroskedasticity pattern, the power of BP test, designed for
detecting heteroskedasticity, is uniformly higher than the other tests.
The visual test power curve is a right shift. This shows a similar story
to the power curves for non-linearity pattern: the conventional test is
more sensitive than the visual test. From the example residual plots at
the bottom we might argue that the heteroskedasticity becomes noticeably
visible around an effect size of 3 or 3.5. However the BP test would
reject at around effect size 2.5. Interestingly, the power curve for the
SW test (for non-normality) is only slightly different to that of the
visual test, suggesting that it performs reasonably for detecting
heteroskedasticity, too. The power curve for the BP test suggests it is
not useful for detecting heteroskeadsticity, as expected.

Overall, the results show that the conventional tests are more sensitive
than the visual test. The conventional tests do have higher power for
the patterns they are designed to detect, but they typically fail to
detect other patterns unless those patterns are particularly strong. The
visual test doesn't require specifying the pattern ahead of time,
relying purely on whether the observed residual plot is detectably
different from ``good'' residual plots. They will perform equally well
regardless of the type of model defect. This aligns with the advice of
experts on residual analysis, who consider residual plot analysis to be
an indispensable tool for diagnosing model problems. What we gain from
using a visual test for this purpose is the removal of any subjective
arguments about whether a pattern is visible or not. The lineup protocol
provides the calibration for detecting patterns: if the pattern in the
data plot cannot be distinguished from the patterns in good residual
plots, then no discernible problem with the model exists.

\hypertarget{comparison-of-test-decisions-based-on-p-values}{%
\subsection{\texorpdfstring{Comparison of test decisions based on
\(p\)-values\label{p-value}}{Comparison of test decisions based on p-values}}\label{comparison-of-test-decisions-based-on-p-values}}

The power comparison demonstrates that the appropriate conventional
tests will reject more aggressively than visual tests, but we don't know
how the decisions for each lineup would agree or disagree. Here we
compare the reject or fail to reject decisions of these tests, across
all the lineups. Figure \ref{fig:p-value-comparison} shows the agreement
of the conventional and visual tests using a mosaic plot for both
non-linearity patterns and heteroskedasticity patterns. For both
patterns the lineups resulting in a rejection by the visual test are
\emph{all} also rejected by the conventional test, except for one from
the heteroskedasticity model. This reflects exactly the story from the
previous section, that the conventional tests reject more aggressively
than the visual test.

For non-linearity lineups, conventional tests and visual tests reject
69\% and 32\% of the time, respectively. Of the lineups rejected by the
conventional test, 46\% are rejected by the visual test, that is,
approximately half as many as the conventional test. There are no
lineups that are rejected by the visual test but not by the conventional
test.

In heteroskedasticity lineups, 76\% are rejected by conventional tests,
while 56\% are rejected by visual tests. Of the lineups rejected by the
conventional test, the visual test rejects more than two-thirds of them,
too.

Surprisingly, the visual test rejects 1 of the 33 (3\%) of lineups where
the conventional test does not reject. Figure \ref{fig:heter-example}
shows this lineup. The data plot in position seventeen displays a
relatively strong heteroskedasticity pattern, and has a strong effect
size (\(log_e(E)=4.02\)), which is reflected by the visual test
\(p\text{-value} = 0.026\). But the BP test \(p\text{-value} = 0.056\),
is slightly above the significance cutoff of \(0.05\). This lineup was
evaluated by 11 participants, it has experimental factors \(a = 0\)
(``butterfly'' shape), \(b = 64\) (large variance ratio), \(n = 50\)
(small sample size), and a uniform distribution for the fitted values.
It may have been the small sample size and the presence of a few
outliers that may have resulted in the lack of detection by the
conventional test.

\begin{figure}

{\centering \includegraphics[width=1\linewidth]{paper_files/figure-latex/p-value-comparison-1} 

}

\caption{Rejection rate ($p$-value $\leq0.05$) of visual test conditional on the conventional test decision on non-linearity (left) and heteroskedasticity (right) lineups (uniform fitted values only) displayed using a mosaic plot. The visual test rejects less frequently than the conventional test, and (almost) only rejects when the conventional test does. Surprisingly, one lineup in the heteroskedasticity group is rejected by the visual test but NOT the conventional test.}\label{fig:p-value-comparison}
\end{figure}

\begin{figure}[t!]

{\centering \includegraphics[width=1\linewidth]{paper_files/figure-latex/heter-example-1} 

}

\caption{The single heteroskedasticity lineup that is rejected by the visual test but not by the BP test. The data plot (position 17) contains a ``butterfly" shape. It visibly displays heteroskedasticity, making it somewhat surprising that it is not detected by the BP test.}\label{fig:heter-example}
\end{figure}

Because the power curve of the visual tests are a shift to the right of
the conventional test (Figures \ref{fig:polypower} and
\ref{fig:heterpower}) we examined whether adjusting the significance
level (to .001, .0001, .00001, \ldots) of the conventional test would
generate similar decisions to that of the visual test. Interestingly, it
doesn't: despite resulting less rejections, neither the RESET or BP
tests come to complete agreement with the visual test (see Appendix).

\hypertarget{effect-of-amount-of-non-linearity}{%
\subsection{\texorpdfstring{Effect of amount of
non-linearity\label{nonlin-analysis}}{Effect of amount of non-linearity}}\label{effect-of-amount-of-non-linearity}}

The order of the polynomial is a primary factor contributing to the
pattern produced by the non-linearity model. Figure
\ref{fig:poly-power-uniform-j} explores the relationship between
polynomial order and power of the tests. The conventional tests have
higher power for lower orders of Hermite polynomials, and the power
drops substantially for the ``triple-U'' shape. To understand why this
is, we return to the application of the RESET test, which requires a
parameter indicating degree of fitted values to test for, and the
recommendation is to generically use four \citep{ramsey1969tests}.
However, the ``triple-U'' shape is constructed from the Hermite
polynomials using power up to 18. If the RESET test had been applied
using a higher power of no less than six, the power curve of
``triple-U'' shape will be closer to other power curves. This
illustrates the sensitivity of the conventional test to the parameter
choice, and highlights a limitation: it helps to know the data
generating process to set the parameters for the test, which is
unrealistic in practice. However, we examined this in more detail (see
Appendix) and found that there is no harm in setting the parameter
higher than four on the tests' operation for lower order polynomial
shapes. Using a parameter value of six, instead of four, yields higher
power regardless of generating process, and is recommended.

For visual tests, we expect the ``U'' shape to be detected more readily,
followed by the ``S'', ``M'' and ``triple-U'' shape. From Figure
\ref{fig:poly-power-uniform-j}, it can be observed that the power curves
mostly align with these expectations, except for the ``M'' shape, which
is as easily detected as the ``S'' shape. This suggests a benefit of the
visual test: knowing the shape ahead of time is \emph{not} needed for
its application.

\begin{figure}

{\centering \includegraphics[width=1\linewidth]{paper_files/figure-latex/poly-power-uniform-j-1} 

}

\caption{The effect of the order of the polynomial on the power of conventional and visual tests. Deeper colour indicates higher order. The default RESET tests under-performs significantly in detecting the "triple-U" shape. To achieve a similar power as other shapes, a higher order polynomial parameter needs to be used for the RESET test, but this higher than the recommended value.}\label{fig:poly-power-uniform-j}
\end{figure}

\hypertarget{effect-of-shape-of-heteroskedasticity}{%
\subsection{\texorpdfstring{Effect of shape of
heteroskedasticity\label{hetero-analysis}}{Effect of shape of heteroskedasticity}}\label{effect-of-shape-of-heteroskedasticity}}

Figure \ref{fig:heter-power-uniform-a} examines the impact of the shape
of the heteroskedasticity on the power of of both tests. The butterfly
shape has higher power on both types of tests. The ``left-triangle'' and
the ``right-triangle'' shapes are functionally identical, and this is
observed for the conventional test, where the power curves are
identical. Interestingly there is a difference for the visual test: the
power curve of the ``left-triangle'' shape is slightly higher than that
of the ``right-triangle'' shape. This indicates a bias in perceiving
heteroskedasticity depending on the direction, and may be worth
investigating further.

\begin{figure}

{\centering \includegraphics[width=1\linewidth]{paper_files/figure-latex/heter-power-uniform-a-1} 

}

\caption{The effect of heteroskedasticity shape (parameter $a$) on the power of conventional and visual tests. The butterfly has higher power in both tests. Curiously, the visual test has slightly higher power for the "left-triangle" than the "right-triangle" shape, when it would be expected that they should be identical, which is observed in conventional testing.}\label{fig:heter-power-uniform-a}
\end{figure}

\hypertarget{effect-of-fitted-value-distributions}{%
\subsection{Effect of fitted value
distributions}\label{effect-of-fitted-value-distributions}}

In regression analysis, predictions are conditional on the observed
values of the predictors, that is, the conditional mean of the dependent
variable \(Y\) given the value of the independent variable \(X\),
\(E(Y|X)\). This is an often forgotten element of regression analysis
but it is important. Where \(X\) is observed, the distribution of the
\(X\) values in the sample, or consequently \(\hat{Y}\), may affect the
ability to read any patterns in the residual plots. The effect of fitted
value distribution on test performance is assess using four different
distributions of fitted values: uniform, normal, discrete and lognormal
(skewed). We expect that if \(\hat{Y}\) has a uniform distribution, it
is easier to read the relationship with the residuals.

\begin{figure}[t!]

{\centering \includegraphics[width=1\linewidth]{paper_files/figure-latex/different-x-dist-poly-power-1} 

}

\caption{Comparison of power on lineups with different fitted value distributions for conventional and visual tests (columns) for non-linearity and heteroskedasticity patterns (rows). The power curves of conventional tests for non-linearity and heteroskedasticity patterns are produced by RESET tests and BP tests, respectively. Power curves of visual tests are estimated using five evaluations on each lineup. For lineups with a uniform fitted value distribution, the five evaluations are repeatedly sampled from the total eleven evaluations to give multiple power curves (grey). Surprisingly, the fitted value distribution has produces more variability in the power of conventional tests than visual tests. Uneven distributions, normal and lognormal distributions, tend to yield lower power.}\label{fig:different-x-dist-poly-power}
\end{figure}

Figure \ref{fig:different-x-dist-poly-power} examines the impact of the
fitted value distribution on the power of conventional (left) and visual
(right) tests for both the non-linearity (top) and heteroskedasticity
(bottom) patterns. For conventional tests, only the power curves of
appropriate tests are shown: RESET tests for non-linearity and BP tests
for heteroskedasticity. For visual tests, more evaluations on lineups
with uniform fitted value distribution were collected, so to have a fair
comparison, we randomly sample five from the 11 total evaluations to
estimate the power curves, producing the multiple curves for the uniform
condition, and providing an indication of the variability in the power
estimates.

Perhaps surprisingly, the visual tests have more consistent power across
the different fitted value distributions: for the non-linear pattern,
there is almost no power difference, and for the heteroskedastic
pattern, uniform and discrete have higher power than normal and
lognormal. The likely reason is that these latter two have fewer
observations in the tails where the heteroskedastic pattern needs to be
detected.

The variation in power in the conventional tests is at first sight,
shocking. However, it is discussed, albeit rarely, in the testing
literature. See, for example, \citet{jamshidian2007study},
\citet{olvera2019relationship} and \citet{zhang2018practical} which show
derivations and use simulation to assess the effect of the observed
distribution of the predictors on test power. The big differences in the
power curves seen in Figure \ref{fig:different-x-dist-poly-power} is
echoed in the results reported in these articles.

\hypertarget{conclusions}{%
\section{Conclusions}\label{conclusions}}

Regression analysis experts suggest that residual plots are
indispensible methods for assessing model fit. We conducted a perceptual
experiment using visual inference to assess this oft-repeated advice.
The experiment tested two primary departures from good residuals:
non-linearity and heteroskedasticity.

We found that conventional residual-based statistical tests are more
sensitive to weak departures from model assumptions than visual tests.
That is, a conventional test concludes there are problems with the model
fit almost twice as often as a human. Conventional tests often reject
when departures in the form of non-linearity and heteroskedasticity are
not visibly different from null residual plots.

While it might be argued that the conventional tests are correctly
detecting small but real effects, this can also be seen as the
conventional tests are rejecting unnecessarily. Many of these rejections
happen even when downstream analysis and results would not be
significantly affected by the small departures from a good fit. The
results from human evaluations provide a more practical solution, which
reinforces the statements from regression experts that residual plots
are an indispensible method for model diagnostics.

It is important to note that residual plots need to be delivered as a
lineup, embedded in a field of null plots. A residual plot may contain
many visual features, but some are caused by the characteristics of the
predictors and the randomness of the error, not by the violation of the
model assumptions. These irrelevant visual features have a chance to be
filtered out by participants with a comparison to null plots, resulting
in more accurate reading. The lineup enables a careful calibration for
reading structure in residual plots.

Human evaluation of residuals is expensive, time-consuming and
laborious. This is possibly why residual plot analysis is often not done
in practice. However, with the emergence of effective computer vision,
our work could lay a foundation for automated residual plot assessment.

The experiment also revealed some interesting results. For the most
part, the visual test performed similarly to the appropriate
conventional test with a shift in the power curve. Unlike conventional
tests, where one needs to specifically test for non-linearity or
heteroskedasticity the visual test operated effectively across the range
of departures from good residuals. If the fitted value distribution is
not uniform, there is a small loss of power in the visual test.
Surprisingly there is a big difference in power of the conventional test
across fitted value distributions. Another surprising finding was that
the direction of heteroskedasticity appears to affect the ability to
visually detect it: both triangles being more difficult to detect than
the butterfly, and a small difference in detection between left- and
right-triangle.

\hypertarget{acknowledgements}{%
\section*{Acknowledgements}\label{acknowledgements}}
\addcontentsline{toc}{section}{Acknowledgements}

These \texttt{R} packages are used for the work: \texttt{cli}
\citep{cli}, \texttt{curl} \citep{curl}, \texttt{dplyr} \citep{dplyr},
\texttt{ggplot2} \citep{ggplot2}, \texttt{jsonlite} \citep{jsonlite},
\texttt{lmtest} \citep{lmtest}, \texttt{mpoly} \citep{mpoly},
\texttt{progress} \citep{progress}, \texttt{tibble} \citep{tibble},
\texttt{ggmosaic} \citep{ggmosaic}, \texttt{purrr} \citep{purrr},
\texttt{tidyr} \citep{tidyr}, \texttt{readr} \citep{readr},
\texttt{stringr} \citep{stringr}, \texttt{here} \citep{here},
\texttt{kableExtra} \citep{kableextra}, \texttt{patchwork}
\citep{patchwork}, \texttt{rcartocolor} \citep{rcartocolor}. The study
website is powered by \texttt{PythonAnywhere} \citep{pythonanywhere} and
the \texttt{Flask} web framework \citep{flask}. The \texttt{jsPsych}
framework \citep{jspsych} is used to create behavioral experiments that
run in our study website.

The article was created with R packages \texttt{rticles}
\citep{rticles}, \texttt{knitr} \citep{knitr} and \texttt{rmarkdown}
\citep{rmarkdown}. The project's GitHub repository
(\url{https://github.com/TengMCing/lineup_residual_diagnostics})
contains all materials required to reproduce this article.

\hypertarget{supplementary-material}{%
\section*{Supplementary material}\label{supplementary-material}}
\addcontentsline{toc}{section}{Supplementary material}

The supplementary material is available at
\url{https://github.com/TengMCing/lineup_residual_diagnostics/blob/master/appendix.pdf}.
It includes more details about the experimental setup, the derivation of
the effect size, the effect of data collection period, and the estimate
of \(\alpha\).

\bibliographystyle{tfcad}
\bibliography{ref.bib}





\end{document}
